\documentclass[float=false, crop=false]{standalone}
\usepackage[subpreambles=true,]{standalone}
% \usepackage{styles/python}
\usepackage{import}

\begin{document}
\section{Associative Arrays and Sets}

\begin{itemize}
	% \require \nameref{sec:ioOperatons}
	% \require \nameref{sec:variablesAndTypes}
	% \require \nameref{subsec:mathematicalOperations}
	\item Indentation Rule
	% \require \nameref{subsec:logicalOperatons} +
	\item Conditional Statements}
	\item Arrays (List and Tuple)
	\item For Loops and Iterations
\end{itemize}


\input{Python/associative_arrays/dictionary}

\input{Python/associative_arrays/for_and_dict}

\subsection{Summary}
In this section we talked about dictionaries, which are one the most powerful built-in types in Python. We learned:
\begin{itemize}[itemsep=1pt]
	\item how to create dictionaries in Python,
	\item methods to alter or manipulate normal and nested dictionaries,
	\item two different techniques for changing an existing \emph{key},
	\item examples on how dictionaries help us organise our data and retrieve them when needed,
\end{itemize}

Finally, we also learned that we can create an \emph{iterable} (discussed in section~\ref{subsec:forLoops}) from dictionary \emph{keys} or \emph{values} using the \pymeth{.key()}, the \pymeth{.values()}, or the \pymeth{.items()} methods.


\input{Python/associative_arrays/frequencyAnalysis}


\include{Python/associative_arrays/set}


\include{Python/associative_arrays/cheatsheet}


\include{Python/associative_arrays/exercises}
\end{document}
